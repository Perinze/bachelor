\section{系统测试}

测试是通过用例和预期结果检验系统行为的关键途径,
能够帮助开发人员快速检测出系统的问题。
在开发人员修改代码片段时,
测试能够快速提供开发人员关心的信息:
程序是否按预期执行?
如果与预期有偏差,具体是哪个行为不符合预期?
有问题的是哪个模块?
如果开发人员有了这些信息,
就能够知道系统是否待完善,
并能够快速定位问题根源,
从而提高软件开发的效率。

\subsection{测试用例的设计原则}

测试用例需要契合系统的接口,
对于本次研究的高层综合流程,
测试用例需要包含输入和输出代码。
输入代码直接作为流程的输入,
而输出代码需要有与预期一致的行为(代码不需要完全一样)。

为了使测试能够尽可能精确地反应系统特定部分的行为
(越精确意味着确定的范围越小),
测试的对象需要尽可能地小。
例如在系统输入相同的情况下,
有两个代码片段都有可能导致系统的同一个错误,
而这两个代码片段分别位于不同且可以分别测试的模块。
如果对整个系统进行该测试,
可能会使开发人员分辨不出来是哪个模块出了问题;
但如果将系统拆分为多个模块,
对这些模块单独测试,
开发人员便会更加轻松地定位到问题代码,
这样无疑是显著提高了测试的效率。

基于上述原则,
本次设计的高层综合流程采取了前端和后端分别测试的策略。
对于前端,测试输入是描述硬件的Python代码,
测试预期结果和输出是对应的XLS IR代码;
对于后端,测试输入是XLS IR代码,
输出则是行为相同的Verilog代码。

\subsection{前端测试}

为了能够完整地测试xlspy的各部分,
需要全面测试所有可能出现的语法,
并保证每个测试单元尽量只反应一种语法的行为。
与此同时,
测试集也需要一些较完整、贴近实际的代码,
既可以反应各模块之间的协同能力,
也可以确定系统能够处理常见规模的输入。

以赋值语句为例,
Python中的赋值语句支持在同一个语句对多个目标变量赋值,
那么编写测试用例时就可以在输入代码中包括多目标赋值,
并检测输出的IR中是否有多条右值相同,
目标不同的赋值语句,
并确保赋值的左右值一一对应。
针对编译器前端的测试用例清单如表\ref{table.2}所示。

\begin{table}[ht]
\begin{center}
\caption{优化流水线的组成表}

\begin{tabular}{ c c }
    \Xhline{3\arrayrulewidth}
    测试用例文件名        & 测试对象 \\
    \hline
    typed\_arg           & 带类型参数、运算符和返回值 \\
    mutable              & 常量、赋值语句以及变量 \\
    multiple\_op         & 赋值和各种运算符 \\
    complete\_branch     & 完整的分支语句、变量捕获 \\
    incomplete\_branch   & 不完整的分支语句、变量捕获 \\
    nested\_branch       & 嵌套分支语句 \\
    for                  & for循环语句 \\
    \Xhline{3\arrayrulewidth}
\end{tabular}

\label{table.2}
\end{center}
\end{table}

\subsection{后端测试}

后端可以拆分为两个部分,
分别是XLS的opt和codegen。
由于本次设计的高层综合流程是基于XLS工具设计的,
在设计测试时,并不需要为依赖的且没有被修改的部分添加测试。
具体而言,
因为我并没有修改codegen部分的代码,
所以并不需要另外编写codegen的测试;
opt除了ir\_parser以及pass中除了TypeExpansionPass,
其它部分也没有测试的必要。

在第\ref{ir_parser}节我们介绍了对ir\_parser的扩展,
为其添加了动态类型的解析规则,
所以这部分的测试需要检查带有动态类型的IR是否被解析为一个类型列表,
且列表里的所有类型与IR一致。
测试TypeExpansionPass时,
则需要检查输入的动态类型IR是否被展开为多个对应类型的模块,
并且有对应的多路复用器根据类型信号选择对应的计算结果。