\section{背景知识}

\subsection{现场可编程门阵列}

现场可编程门阵列(Field Programmable Gate Array,FPGA)是一种半导体器件,
由可编程、可互联的可配置逻辑块(CLB)阵列构成。
FPGA可以重新编程以适应不同需求的应用场景和功能要求,
有着特定用途集成电路(ASIC)无法比拟的灵活性,
在各种特殊应用集成电路中脱颖而出,
现已广泛应用于各种领域,
包括数字信号处理、图像和视频处理、汽车、高性能计算和数据处理等领域。

FPGA能够进行的各种逻辑运算是由阵列中大量的可配置逻辑块完成的。
每个可配置逻辑块包含一组逻辑门、触发器等数字系统原件。
逻辑门可以被编程以执行各种逻辑操作,如布尔运算、逻辑组合;
而触发器则用于存储状态信息,通常用于实现时序逻辑或状态机,
在时序设计中起到重要作用。
除了可配置逻辑块的配置,
还需要互连资源作为可配置逻辑块之间的连接方式。
互连资源包括内部信号线、可编程开关和互连点,
这些资源在可配置逻辑块之间建立连接,
实现不同的信号传输路径。

这些配置信息存储在FPGA的配置存储器中,
通常是RAM。
当FPGA上电或者重新配置时,
配置存储器中的信息就会被加载到可配置逻辑块中,
从而确定可编程逻辑块的功能的连接方式。
这些配置信息在FPGA开发中常常被称为比特流,
由硬件描述语言综合器生成,
用于加载到FPGA的RAM中。

\subsection{硬件描述语言}

硬件描述语言(Hardware Description Language,HDL)是一类用于描述数字电路系统的描述语言。
它允许硬件工程师摆脱CAD工具,
转而通过编写代码来描述电路的行为和结构,
从而方便地设计、验证和仿真电子系统。
几乎所有FPGA开发都离不开硬件描述语言,
因为它是将设计概念转化为可执行硬件的关键工具之一。

最常见的硬件描述语言是VHDL(Very High Speed Integrated Circuit HDL)
和Verilog,
允许硬件开发人员以结构化的方式描述硬件功能和行为。
这些语言看起来非常像软件编程语言,
但它们实际上并不会以软件编程语言的执行模型,即通用计算架构的形式运行。
数字系统是由数字电子元件(如逻辑门、触发器等)组成的系统,
用于处理数字信号,
而硬件描述语言实际上就是数字系统的描述语言,
他的执行模型通常不会出现自动变量、函数调用等概念,
而是逻辑运算、算术运算、模块例化,以及这些单元之间的连接方式,
这正好对应了数字系统中的模块以及连接线。

硬件描述语言与软件开发语言在编写时最大的不同就是并非顺序执行。
以Verilog为例,
对于同一个模块下的连续赋值语句、always块或者其他模块的例化,
并不能根据这些语句所在行的先后顺序判断它们的计算顺序,
而是要根据这些语句之间的关联推断出这些计算的依赖关系。

并非顺序执行的另一层含义是并行性。
Verilog描述的硬件在计算过程中,
并不是阻塞的,而是所有模块都会同时地、并行地、连续地进行计算。
比如要描述一个简单的算术$a + (b * c)$,
那么加法器和乘法器任何时刻都在进行计算,
不存在二者只有其一计算或者二者都在等待的情况。

等待的情况也是存在的,即时序逻辑。
数字系统中的时序逻辑会随着时钟信号的变化采取行动,
例如在时钟上升沿加载所有D触发器。
时序逻辑在数字系统的设计中非常重要,
因为数字系统的正确操作往往需要考虑信号的时序关系,
即信号的产生、传输和相应之间的时间顺序,
这是仅仅通过组合逻辑无法处理的。
另外,在FPGA设计中,
时序逻辑可以帮助设计者满足时序约束,
确保电路在特定的时钟周期内完成所需的操作,
以避免时序冲突和时序失败。

还有一种硬件描述语言结合了仿真的功能,
称为硬件描述和仿真语言,
如SystemVerilog。
使用这种语言可以更方便地进行验证和调试。
SystemVerilog还提供了一些面向对象的特性,
如类和继承,可以帮助管理复杂的设计和测试环境。
相比传统的通过原理图设计,
手工布局布线并设计控制逻辑的硬件开发流程,
硬件描述语言为硬件工程师带来了极大便利。

\subsection{高层综合技术}

高层综合(High-Level Synthesis,HLS)技术的兴起源于数字系统设计领域复杂性不断增加,
手动编写硬件描述语言所耗费的时间和精力越来越多,并且容易出错。
同时,高级编程语言的广泛应用提高了人们从事开发时思考的抽象级别,
软件开发人员能够更容易地表达和实现复杂的算法和功能。
然而,将高级程序代码转换为硬件电路通常需要深入了解硬件设计的知识和技能,
这对于大多数软件开发人员来说是一项挑战。
因此,高层综合技术的出现旨在提高硬件设计的生产力和效率。
通过将高级程序代码转换为硬件描述语言,
设计者可以利用高级编程语言的优势,
如模块化、抽象和重用,从而更快地实现复杂的硬件功能。

此外,许多应用对于处理速度的要求越来越高,
传统的软件实现无法满足这种需求。
高层综合技术可以将部分或全部软件算法转换为硬件电路,
从而实现硬件加速,提高系统的性能和效率。

综上所述,高层综合技术的发展是为了应对数字系统设计的复杂性增加、
抽象级别提高、设计生产力提高、应用需求多样化以及硬件加速需求等挑战。
它为设计者提供了一种自动化的方法来将高级程序代码转换为硬件电路
从而加速硬件设计过程并提高系统性能和可扩展性。

高层综合编译器的首先需要对高级语言进行前端分析,
将输入代码转换为内部表示形式,即抽象语法树,
再进行一些后续处理,
如类型推导和中间表示生成。

中间表示(Intermediate Representation,IR)在编译器架构中不是必须的,
但现代编译器设计往往需要引入中间表示来简化编译器设计。
中间表示独立于源语言和目标语言,
而且可以将源语言和目标语言之间的转化过程分解为多个独立的阶段,
使得编译器的设计和实现更加模块化,易于扩展和维护。
中间表示的存在也为编译器提供了一个优化和分析的平台,
因为中间表示往往是精心为优化和分析设计的,
如LLVM为其IR提供了齐全的优化分析基础设施。
常见优化包括常量折叠、死代码消除、循环优化等,
而常见分析则包括数据流分析、控制流分析、依赖分析等。

以上对于中间表示的处理以不改变代码语义为前提,
而中间表示也提供一种方便的方式进行程序变换。
编译器可以对中间表示进行各种变换,
如代码重排、函数内联、循环展开等,
这些变换会潜在地改变程序的行为,
但仍保证在绝大多数情况下获得正确的结果,
因为极端情况极少会出现,
且往往不是程序期望的输入,
可以通过对输入进行检查来避免。

\subsection{动态类型以及类型推导}

动态类型语言是一种在运行时进行类型检查的编程语言,
其中变量的类型是动态确定的,
而不是在编译时确定的。
动态类型语言编写的程序中,
变量的类型可以随着程序的执行而改变,
并且通常不需要显式地声明变量的类型。
常见的动态类型语言包括Python、JavaScript、Ruby等。

动态类型语言通常具有简介的语法和表达能力,
使得使用动态类型语言开发程序非常简单直观,
且灵活性较高,
非常适用于快速原型开发和脚本编程等场景,
包括敏捷的Web开发、数据分析、科学计算等。

但事物的存在往往具有两面性。
由于动态类型语言的变量类型是在运行时动态确定的,
这会导致类型不确定性,
因而动态类型语言往往不会用于开发大型稳定的系统。
另外,这样的灵活性也导致动态类型语言在硬件描述领域少见,
因为硬件描述语言要求变量的类型在编译时期确定。
如果不对动态类型语言的类型可能性加以限制,
就无法根据动态类型语言生成对应的硬件描述语言代码。
硬件描述语言也需要严格管理硬件资源(如寄存器、存储器、逻辑单元等),
而动态类型语言不提供对硬件资源的直接管理,
这也为动态类型语言的高层综合带来的困难。

\subsection{浮点数以及IEEE 754规范}

浮点数是一种用于表示实数(包括小数和整数部分)的数学表示方法,
相比于整数和定点数,
浮点数牺牲了一些确定性和性能,
换取了更广的表示范围和灵活性,
通常用于表示小数、非常大或非常小的数。

浮点数是采用科学计数法的方式来表示,
它由3个部分组成,
分别是符号*S*、指数*E*和尾数*M*,
其值可以这样定义

$$V =  (-1)^S * M * 2^E$$

在使用浮点数时,
需要确定E和M的位数,
不同的位数分别可以决定不同的数据范围和精度。
目前最广泛使用的浮点数标准是IEEE754,
规定了单精度浮点数(float)和双精度浮点数(double)的浮点格式。

IEEE 754是一种浮点数规范,
它定义了浮点数的表示方法、运算规则以及舍入规则等。
在IEEE 754的规定下,
由于尾数M的第一位总是1(因为1 <= M < 2),
所以这个1需要省略,
所以单精度的23位尾数可以表示24位有效数字,
双精度的52位尾数可以表示53位有效数字。
而指数E是一个无符号整数,
单精度情况下占8位,
所以其取值范围是0~255。
但因为指数可以为负数,
所以规定在存入E时要在其原本的值上加一个中间数127,
所以E的取值范围其实是-127~128。
双精度浮点数同理,11位的指数E,
加上中间数1023后取值范围为-1023~1024。
IEEE 754规范还定义了特殊的浮点数值,
如正负无穷大(±∞)、NaN(Not a Number)等,
以及对这些特殊值的处理规则。

除了数字本身的表示方法,
IEEE 754标准还定义了浮点数的基本运算规则,
包括加法、减法、乘法和除法等。
在进行浮点数运算时,
需要考虑到舍入误差、溢出和下溢等问题,
以确保计算结果的正确性和精度。