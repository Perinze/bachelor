\begin{abstract}

近年来,
随着世界各地的计算需求日益增长,
基于数字系统的硬件加速器复杂性也不断增加,
导致传统的硬件设计方法变得越来越繁琐和耗时。
在这种情况下,高层综合(High-Level Synthesis,HLS)技术成为了一种强大的工具,
它能够将高级程序代码(如C、C++等)转换为硬件描述语言(如VHDL或Verilog),
从而加速了数字系统的设计和开发过程。
Python作为一种动态类型语言,
其简洁的语法和丰富的库使得它在软件开发领域中备受欢迎。
然而,由于Python的动态类型特性以及与硬件描述语言的差异,
将Python代码直接应用于硬件描述领域面临诸多挑战。

本论文旨在探讨如何将Python高层综合技术应用于数字系统的设计和开发中。
首先,将介绍高层综合技术以及其他相关技术的基本原理和发展历程,
以及其在数字系统设计领域的重要性和应用价值。
然后,将对Python语言的特点进行分析,
探讨其与硬件描述语言的差异和相似之处,
以及将Python应用于高层综合的可行性和优势。

接下来,将介绍本文提出的基于XLS编译器开发的Python高层综合编译器,
包括其设计原理、功能特性、实现方法等,
并介绍该编译器如何生成带有浮点数的寄存器传输级代码。
最后,将通过实验和案例分析,
验证Python高层综合编译器在数字系统设计中的有效性和实用性,
并展望未来的发展方向和应用前景。

通过对Python高层综合技术的研究和探讨,
可以为数字系统设计领域提供新的思路和方法,
促进数字系统的快速设计和开发,
提高系统的性能和可靠性,
推动数字系统设计领域的进步和发展。

\keywords{现场可编程门阵列;硬件加速;高层综合;动态类型语言;编译器}
\end{abstract}

\begin{enabstract}
As computing demands around the world continue to grow,
the complexity of hardware accelerators based on digital systems is also increasing.
This trend has led to traditional hardware design methods becoming increasingly cumbersome and time-consuming.
In this context, High-Level Synthesis (HLS) technology has emerged as a powerful tool 
that can convert high-level program code (such as C, C++) 
into hardware description languages (such as VHDL or Verilog), 
thereby accelerating the design and development process of digital systems. 
Python, as a dynamic type language, 
with its concise syntax and rich libraries, 
is widely popular in the field of software development. 
However, due to the dynamic type nature of Python 
and the differences from hardware description languages, 
applying Python code directly to the hardware description domain faces many challenges.

This paper aims to explore how Python High-Level Synthesis technology 
can be applied to the design and development of digital systems. 
Firstly, the basic principles and development history of High-Level Synthesis technology 
will be introduced, as well as its importance and application value 
in the field of digital system design. 
Then, an analysis of the characteristics of the Python language will be conducted, 
exploring its differences and similarities with hardware description languages, 
as well as the feasibility and advantages of applying Python to High-Level Synthesis. 

Next, we will introduce the Python High-Level Synthesis compiler developed 
based on the XLS compiler proposed in this paper, 
including its design principles, features, implementation methods, etc,
Additionally, we will discuss how our compiler generates register transfer level (RTL) 
code with floating-point numbers.
Finally, through experiments and case studies, 
we will verify the effectiveness and practicality of the Python High-Level Synthesis 
compiler in digital system design, 
and outline future development directions and application prospects.

Through the research and discussion of Python High-Level Synthesis technology, 
new ideas and methods can be provided for the field of digital system design, 
promoting the rapid design and development of digital systems, 
improving system performance and reliability, 
and driving progress and development in the field of digital system design.

\enkeywords{FPGA; Hardware Accelerating; High-Level Synthesis; Dynamic }
\end{enabstract}